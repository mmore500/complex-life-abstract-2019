\begin{abstract}

Natural evolutionary history is profoundly shaped by evolutionary transitions in individuality, episodes where independent replicating entities united to form more complex replicating entities.
Fraternal transitions in individuality occur specifically when the derived replicating entity is composed of kin groups of a lower-level replicator;
examples include the evolution of multicellularity and the evolution of eusocial insect colonies.
The conditions necessary for fraternal transitions to occur and the mechanisms by which such transitions occur have been a fruitful target of scientific interest and warrant continued investigation into many interesting open questions.
Digital evolution techniques allow pursuit of biologically-motivated questions that otherwise, \textit{in vivo}, might require experiments that are slow, expensive, or call for manipulation of physically immutable variables. %

We present an extension of previous exploratory work with the DISHTINY (Distributed Hierarchical Transitions in IndividualitY) platform.
In this current work we introduce a more open-ended digital organism model in which cells are controlled by heritable computer programs, similarly to the well-known Avida digital evolution system.
Specifically, we incorporate the SignalGP genetic programming representation, which allows for dynamic (event-driven) interactions among digital cells and between cells and their environment.
DISHTINY allows SignalGP-controlled cells to unite into explicitly registered cooperating groups.
Membership in a cooperating group is limited to kin: at birth, cells can either join the same cooperative network as its parent or begin a new network.
A cooperative resource-collection mechanism incentivizes group formation.
DISHTINY cells evolve phenotypes characteristic of fraternal transitions in individuality with respect to these cooperating groups: reproductive division of labor, resource sharing (including, in some replicates, endowment of offspring propagule groups), and apoptosis.
% strategic apoptosis, but we don't  quite have complete evidence of this yet

Our research raises a slew of exciting potential extensions.
DISHTINY's fundamentally spatial and distributed conception allows high-performance parallel computing hardware to be leveraged to perform large-scale multicellularity experiments consisting of millions, instead of thousands, of cells.
The dynamic, event-driven design of SignalGP provides avenues to introduce ploidy and enable direct communication and resource sharing between distant cells (e.g., neurological and vascular faculties).
From an applied perspective, fraternal transitions in individuality might yield more complex digital organisms that exhibit more sophisticated, and potentially useful, capabilities, for example, as reinforcement-learning agents.
From a scientific perspective, we believe that open-ended digital multicellular systems will provide insight into evolutionary transitions in individuality that is unique from, but complementary to, traditional experimental and observational methods.

\end{abstract}
