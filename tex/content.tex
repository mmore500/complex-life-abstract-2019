\begin{abstract}

Natural evolutionary history is profoundly shaped by evolutionary transitions in individuality, episodes where independent replicating entities united to form more complex replicating entities.
Among such events are fraternal transitions in individuality, where the the derived replicating entity is composed of kin groups of a lower-level replicating entity.
Examples include the evolution of multicellularity or the evolution of eusocial insect colonies.
The conditions necessary for fraternal transitions in individuality to occur and the mechanisms by which such transitions occur have long been a fruitful target of scientific interest, and warrant continued work to further develop and test our understandings of these phenomena.
Digital evolution techniques allow pursuit biologically-motivated questions through experiments that might otherwise be slow, expensive, or require manipulation of physically-impossible variables.
From an applied perspective, fraternal transitions in individuality in digital systems might yield more complex digital organisms with more sophisticated, and potentially useful, capabilities.

We present an extension of previous exploratory work with the DISHTINY (Distributed Hierarchical Transitions in IndividualitY) platform.
In this work, we incorporate the SignalGP genetic programming representation, which allows for dynamic, event-driven interactions among digital cells and between cells and their environment.
Through a cooperative resource-collection mechanism, DISHTINY incentivizes kin groups of SignalGP-controlled cells to unite into self-replicating collectives.
We exploit SignalGP to introduce a novel approach in grid-based cooperation and coordination digital evolution work where genetic programming is mirrored across several computational instances to facilitate the evolvability of directionally-symmetric cellular phenotypes.
In evolution experiments we observe phenotypes characteristic of fraternal transitions in individuality: reproductive division of labor, resource-sharing (including, in some replicates, endowment of offspring propagule groups), and cell suicide.

This work begs a slew of exciting extensions.
We aim to exploit DISHTINY's fundamentally spatial and distributed conception to harness high-performance parallel computing hardware to perform large scale multicellularity experiments.
We are also interested in exploiting the dynamic, event-driven design of SignalGP to introduce ploidy and computationally performant overlay of direct, spatially-shortcutting neuro/vascular faculties.
Ultimately, we hope to develop large-scale digital multicellularity as a substrate to evolve reinforcement-learning agents with an eye towards tasks pursued in existing and emerging AI research, such as the Open AI Gym and the Animal AI Olympics.

\end{abstract}
