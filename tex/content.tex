\begin{abstract}

Natural evolutionary history is profoundly shaped by evolutionary transitions in individuality, episodes where independent replicating entities united to form more complex replicating entities.
Fraternal transitions in individuality occur specifically when the derived replicating entity is composed of kin groups of a lower-level replicating entity;
examples include the evolution of multicellularity and the evolution of eusocial insect colonies.
The conditions necessary for fraternal transitions in individuality to occur and the mechanisms by which such transitions occur are a fruitful target of scientific interest and warrant continued investigation.
Digital evolution techniques allow pursuit of biologically-motivated questions that otherwise, \textit{in vivo}, might be slow, expensive, or call for manipulation of physically immutable variables.
From an applied perspective, fraternal transitions in individuality might yield more complex digital organisms that exhibit more sophisticated, and potentially useful, capabilities.

We present an extension of previous exploratory work with the DISHTINY (Distributed Hierarchical Transitions in IndividualitY) platform.
In this current work, we incorporate the SignalGP genetic programming representation, which allows for event-driven interactions among digital cells and between cells and their environment.
In order to facilitate evolvability of directionally-symmetric phenotypes, we exploit the dynamic nature of SignalGP to mirror genetic programming across directionally-arrayed computational instances, a novel innovation in grid-based digital evolution work.
Through a cooperative resource-collection mechanism, DISHTINY incentivizes SignalGP-controlled cells to unite into explicitly registered self-replicating kin collectives.
In evolution experiments we observe phenotypes characteristic of fraternal transitions in individuality: reproductive division of labor, resource-sharing (including, in some replicates, endowment of offspring propagule groups), and cell suicide.

Our research begs a slew of exciting extensions.
High-performance parallel computing hardware can exploit DISHTINY's fundamentally spatial and distributed conception to enable large-scale multicellularity experiments consisting of millions, instead of thousands, of cells.
The dynamic, event-driven design of SignalGP provides avenues introduce ploidy and computationally performant overlay of direct, spatially-shortcutting neurological and vascular faculties.
Ultimately, we hope to develop large-scale digital multicellularity as a substrate to evolve reinforcement-learning agents with an eye towards tasks pursued in existing and emerging AI research, such as the Open AI Gym and the Animal AI Olympics.

\end{abstract}
