\begin{abstract}

Natural evolutionary history includes episodes where independent replicating entities united to form more complex replicating entities.
Among these events are fraternal transitions, such as the evolution of multicellularity or the evolution of eusocial insect colonies, where the derived replicating entity is composed of kin groups of a lower-level replicating entity.
The necessary conditions for these events to occur and the mechanisms by which they occur have been a target of scientific interest, and merit continued work further developing and testing our understanding of these phenomena.
Digital evolution provides an opportunity to perform experiments that might otherwise be difficult/expensive or impossible in a setting where complete knowledge and replayability is possible.
From an applied perspective, instantiating fraternal transitions in individuality in digital systems the evolution of more complex --- and potentially real-world useful --- digital organisms.

We extend previous exploratory work with the DISHTINY (Distributed Hierarchical Transitions in IndividualitY) platform, introducing the new SignalGP genetic programming representation.
SignalGP allows for dynamic, event-driven interactions among cells and between cells and their environment.
Through a cooperative resource-collection mechanism, DISHTINY incentivizes kin groups of SignalGP-controlled cells to unite into a meta-replicating group.
We exploit SignalGP to introduce a novel approach in grid-based cooperation and coordination digital evolution work where genetic programming is mirrored across several computational instances to facilitate the evolvability of directionally-symmetric cellular phenotypes.
In evolution experiments we observe the following phenotypes among digital cells: reproductive division of labor, resource-sharing (including endowment of offspring propagule groups), and cell suicide.
This work prompts a slew of promising extensions such as exploiting the DISHTINY platform's fundamentally distributed conception to harness high-performance parallel computing hardware for aggressive scale (e.g., excascale) experiment and exploiting SignalGP to introduce ploidy and computationally-performant overlay of evolving spatially direct linking neuro/vascular faculties.
Ultimately, we hope to explore large-scale digital multicellularity as a substrate for reinforcement-learning agents using tasks from existing and emerging research, such as the Open AI Gym and the Animal AI Olympics.

\end{abstract}
